\documentclass{article} % тип документа (book, report)

%  Русский язык
\usepackage[warn]{mathtext}         % кирилица в формулах
\usepackage[T2A]{fontenc}			% кодировка
\usepackage[utf8]{inputenc}			% кодировка исходного текста
\usepackage[english,russian]{babel}	% локализация и переносы
\usepackage[thinlines]{easytable}   % таблицы

\usepackage{amsmath,amsfonts,amssymb,amsthm,mathtools} % Математика
\usepackage{wasysym}   % Шрифты
\usepackage{graphicx}  % графика
%\usepackage{fancyhdr} % Колонитулы
%\pagestyle{fancy}     % Колонитулы
\usepackage{xcolor}    % цвет фона
\usepackage[paper=portrait,pagesize]{typearea} % горизонтальный лист
\usepackage{pdflscape}
\usepackage{lscape}

\pagecolor{black}
\color{white}

%Заговолок
\title{МЦНМО. КЭМ. Уравнения n-степени.}
\author{Бодан Нелимов}
\date{\today}


\begin{document} % начало документа
	
	\maketitle
	\newpage
	
	\section{Формула для вычисления корней квадратного уравнения}{
		Чтобы не мучаться с $3$ коэфицентам пока возьму только два.\\
		$x^2+bx+c=0$\\
		$x^2+2\frac{bx}{2}+c=0$\\
		$x^2+2\frac{bx}{2}+(\frac{b}{2}^2-\frac{b}{2}^2)+c=0$\\
		$(x+\frac{b}{2})^2-\frac{b}{2}^2+c=0$\\
		$(x+\frac{b}{2})^2-\frac{b^2}{4}+\frac{4c}{4}=0$\\
		$(x+\frac{b}{2})^2-\frac{b^2-4c}{4}=0$\\
		$(x+\frac{b}{2})^2-\frac{\sqrt{b^2-4c}}{2}^2=0$\\
		$(x+\frac{b}{2}-\frac{\sqrt{b^2-4c}}{2})(x+\frac{b}{2}+\frac{\sqrt{b^2-4c}}{2})=0$\\\\
		$x_{1,2} = \frac{-b\pm\sqrt{b^2-4c}}{2}$\\\\
		Теперь добавлю коэфицент $a$ и сделаю то же самое:\\
		$ax^2+bx+c=0$\\
		$a(x^2+\frac{bx}{a}+\frac{c}{a})=0$\\
		$a(x^2+2\frac{bx}{2a}+\frac{c}{a})=0$\\
		$a(x^2+2\frac{bx}{2a}+(\frac{b}{2a}^2 - \frac{b}{2a}^2)+\frac{c}{a})=0$\\
		$a((x+\frac{b}{2a})^2 - \frac{b^2}{4a^2} + \frac{4ac}{4a^2})=0$\\
		$a((x+\frac{b}{2a})^2 - \frac{b^2-4ac}{4a^2})=0$\\
		$a((x+\frac{b}{2a})^2 - \frac{\sqrt{b^2-4ac}}{2a}^2)=0$\\
		$a(x+\frac{b}{2a} - \frac{\sqrt{b^2-4ac}}{2a})(x+\frac{b}{2a} + \frac{\sqrt{D}}{2a})=0$\\\\
		$x_{1,2} = \frac{-b\pm\sqrt{b^2-4ac}}{2a} = \frac{-b\pm\sqrt{D}}{2a}$\\
		Получили привичную со школы формулу для корней квадратного уравнения. Потом попробую вывести для остальных степеней, я знаю что они все уже найдены кем-то, просто ради интереса, хочу пройти этот путь.
	}
	\section{Теорема Виета для уравнений $n$ степени}{
		Замечу что это уравнение было получено преобразованием без изменения значений правой и левой части:\\
		$$(x+\frac{b}{2}-\frac{\sqrt{b^2-4c}}{2})(x+\frac{b}{2}+\frac{\sqrt{b^2-4c}}{2})=0$$\\
		То есть эквивалентом записи полинома $2$ степени будет:\\
		$$(x-x_1)(x-x_2) = 0$$\\
		Приведу эту запись в привычный вид:\\
		$x^2-xx_1-xx_2+x_1x_2 = 0$\\
		$x^2-xx_1-xx_2+x_1x_2 = 0$\\
		$x^2-(x_1+x_2)x+x_1x_2 = 0$\\
		$x^2+k_1x+k_0=0$\\
		Для удобства буду обозначать коэфиценты через $k$. Если учитывать что мы делали преобразование без изменения значений, то можно соотнести коэфиценты и составить такую систему уравнений:\\
		$$\begin{cases} x_1+x_2 = -k_1\\ x_1x_2 = k_0\end{cases}$$\\
		Проделаю то же самое для $3$ и $4$ степени уравнения:\\
		$(x-x_1)(x-x_2)(x-x_3) = 0$\\
		$(x^2-xx_1-xx_2+x_1x_2)(x-x_3) = 0$\\
		$(x^3-x^2x_1-x^2x_2+xx_1x_2-x^2x_3+xx_1x_3+xx_2x_3-x_1x_2x_3)=0$\\
		$(x^3-x^2(x_1+x_2+x_3)+x(x_1x_2+x_1x_3+x_2x_3)-x_1x_2x_3)=0$\\
		$x^3-x^2(x_1+x_2+x_3)+x(x_1x_2+x_1x_3+x_2x_3)-x_1x_2x_3)=0$\\
		$x^3+k_2x^2+k_1x+k_0 = 0$\\
		$$\begin{cases} x_1+x_2+x_3 = -k_2\\ x_1x_2+x_1x_3+x_2x_3 = k_1\\ x_1x_2x_3=-k_0\end{cases}$$\\
		$(x-x_1)(x-x_2)(x-x_3)(x-x_4) = 0$\\
		$x^4-x^3(x_1+x_2+x_3+x_4)+x^2(x_1x_2+x_1x_3+x_2x_3+x_1x_4+x_2x_4+x_3x_4) - \\ -x(x_1x_2x_3+x_1x_2x_4+x_1x_3x_4+x_2x_3x_4)+x_1x_2x_3x_4 = 0$\\
		$x^4+k_3x^3+k_2x^2+k_1x+k_0 = 0$\\
		$$\begin{cases} x_1+x_2+x_3+x_4 = -k_3\\ x_1x_2+x_1x_3+x_2x_3+x_1x_4+x_2x_4+x_3x_4 = k_2\\ x_1x_2x_3+x_1x_2x_4+x_1x_3x_4+x_2x_3x_4= -k_1\\  x_1x_2x_3x_4= k_0\end{cases}$$\\
		Видна закономерность. Уравнение $n$ степени выглядит так:\\
		$$\prod_{j=1}^{n}x-x_j = (x-x_1)(x-x_2)\cdots(x-x_{n-1})(x-x_n) = 0$$
		Коэфицент $k_m$ при $x^m$ состовляется из суммы всех членов умноженных на $m$ количество $x$. То есть $k_m$ это сумма всех способов выбрать произведение $m$ количества корненей. Так как это задача комбинаторики, то количество членов в $k$ строке системы уравнений Виета для $n$ степени это биномиальный коэфицент ${n \choose k}$. Для удобного описания этого на языке математики введу функцию $S(k, n)$, где $n$ степень уравнения, $k$ - номер строки в системе. То есть вот так выглядит система Виета для уравнения $n$ степени:
		$$Polynom(n) = x^n + \sum_{j=0}^{n-1}k_jx^j = x^n+k_{n-1}x^{n-1}+\cdots+k_1x+k_0 = 0$$
		
		$$
		\begin{dcases}
			S(1, n) = -k_{n-1}\\
			S(2, n) = k_{n-2}\\
			\vdots\\
			S(k, n) = (-1)^{k}k_{n-k}\\
			\vdots\\
			S(n-1, n) = (-1)^{n-1}k_1\\
			S(n, n) = (-1)^{n}k_0\\
		\end{dcases}
		$$
		
		А вот так выглядит обобщение для функции $S(k, n)$:
		\begin{align*}
			&S(1, n) = \sum_{(j_1=1)}^{n}x_{j_1}\\
			&S(2, n) = \sum_{(j_1=1)}^{n}\sum_{(j_2=j_1+1)}^{n}x_{j_1}x_{j_2}\\
			&S(3, n) = \sum_{(j_1=1)}^{n}\sum_{(j_2=j_1+1)}^{n}\sum_{(j_3=j_2+1)}^{n}x_{j_1}x_{j_2}x_{j_3}\\
			\vdots\\
			&S(k, n) = \sum_{(j_1=1)}^{n}\sum_{(j_2=j_1+1)}^{n}\cdots\sum_{(j_{k-1}=j_{k-2}+1)}^{n}\sum_{(j_{k}=j_{k-1}+1)}^{n}x_{j_1}x_{j_2}\cdots x_{j_{k-1}}x_{j_k} = \sum_{1\leq j_1 \leq\cdots\leq j_k \leq n}^{n}\left(\prod_{i=1}^{k}{x_{j_i}}\right)\\
			\vdots\\
			&S(n, n) = \prod_{j=1}^{n}{x_j}
		\end{align*}	
	}
\end{document} % конец документа