\documentclass{article} % тип документа (book, report)

%  Русский язык
\usepackage[warn]{mathtext}         % кирилица в формулах
\usepackage[T2A]{fontenc}			% кодировка
\usepackage[utf8]{inputenc}			% кодировка исходного текста
\usepackage[english,russian]{babel}	% локализация и переносы
\usepackage[thinlines]{easytable}   % таблицы

\usepackage{amsmath,amsfonts,amssymb,amsthm,mathtools} % Математика
\usepackage{wasysym}   % Шрифты
\usepackage{graphicx}  % графика
%\usepackage{fancyhdr} % Колонитулы
%\pagestyle{fancy}     % Колонитулы
\usepackage{xcolor}    % цвет фона
\usepackage[paper=portrait,pagesize]{typearea} % горизонтальный лист
\usepackage{pdflscape}
\usepackage{lscape}

\pagecolor{black}
\color{white}

%Заговолок
\title{МЦНМО. КЭМ. Конечные суммы k-степени, d-порядка.}
\author{Богдан Нелимов}
\date{\today}


\begin{document} % начало документа
	
	\maketitle
	\newpage
	
	\section{Введение}{Я заинтересовался конечными суммами натурального ряда от 1 до n, степени k. Обозначим такую сумму как S: 
		$$S_k(n) = \sum_{j=1}^{n}j^k = 1^k + 2^k + \cdots + n^k$$
		Выпишем аналитическое представление для первых нескольких k:
		\begin{align*}
			S_0(n) &= n\\
			S_1(n) &= \dfrac{n(n+1)}{2}\\
			S_2(n) &= \dfrac{n(n+1)(2n+1)}{6}\\
			S_3(n) &= \dfrac{n^2(n+1)^2}{4} = S_1(n)\\
			S_4(n) &= \dfrac{n(n+1)(2n+1)(3n^2 + 3n - 1)}{30}\\
			S_5(n) &= \dfrac{n^2(n+1)^2(2n^2+2n-1)}{12}\\
			S_6(n) &= \dfrac{n(n+1)(2n+1)(3n^4+6n^3-3n+1)}{42}\\
			S_7(n) &= \dfrac{n^2(n+1)^2(3n^4+6n^3-n^2-4n+2)}{24}\\
			\vdots
		\end{align*}
		Если взять $S_k(n)$ но к общему члену добавить 1 под степень, а потом отнять из него ряд $S_k(n)$, то получится красивая телескопическая сумма в которой останется только первый и последний элементы ряда. Если записать это как равенство, то из него можно вывести формулу для $S_{k-1}$, можете попробвать вывести $S_1$ подставив k=2:
		$$\sum_{j=1}^{n}(j+1)^k - j^k = (n+1)^k - 1^k$$
		Попробую привести в более явному виду. Для начал распишем общий член суммы по биному ньютона:
		$$(j+1)^k - j^k = \sum_{t=0}^{k-1}{{k}\choose{t}}j^t = {{k}\choose{1}}j^{k-1} + {{k}\choose{2}}j^{k-2} + \ldots + {{k}\choose{k-1}}j^{1} + 1$$
		И теперь всё становится проще:
		
		$$\sum_{j=1}^{n}\sum_{t=0}^{k-1}{{k}\choose{t}}j^t = (n+1)^k - 1^k$$
		
		$${{k}\choose{1}}\sum_{j=1}^{n}j^{k-1} + {{k}\choose{2}}\sum_{j=1}^{n}j^{k-2} + \ldots + {{k}\choose{k-1}}\sum_{j=1}^{n}j + \sum_{j=1}^{n}1 = (n+1)^k - 1^k$$
		
		$${{k}\choose{1}}S_{k-1} + {{k}\choose{2}}S_{k-2} + \ldots + {{k}\choose{k-1}}S_{1} + S_{0} = (n+1)^k - 1^k$$
		
		$$S_{k-1} = \dfrac{1}{k}((n+1)^k - 1^k - \sum_{t=2}^{k}{{k}\choose{t}}S_{k-t})$$
		Красота. Кажется это называется многочленом Бернулли и как-то связано с числами Бернулли. Я плохо в этом разбираюсь, но хочу разобраться. Такая рекурентная формула хоть и даёт нам более быстрый способ, но для больших k считать придётся очень долго. Также не понятно как обобщать формулу для дробных и отрицательных k. 
		
		В введении я рассматривал то что я нашёл сам, но потом обнаружил что это было уже найдено до меня, дальше я буду писать о своих идеях обобщения, которые я не видел у других. Честно говоря я специально не гуглил, чтобы не огорчаться от своей вторичности, как допишу работу тогда и узнаю - я просто повторяю успехи предшественников, или делаю какой-то настоящий вклад.
	}
	
	\section{Суммы d-порядка для k = 1}{Я мыл посуду и мне стало интересно, как найти сумму сумм натурального ряда от 1 до n, а также сумму сумм сумм натурального ряда от 1 до n и так далее :) Попробуем же найти:
		$$\sum_{j=1}^{n}S_1(j) = \sum_{j=1}^{n}\dfrac{j(j+1)}{2} = \dfrac{1}{2}\sum_{j=1}^{n}j^2+j = \dfrac{S_2+S_1}{2} = \dfrac{1}{2}(\dfrac{n(n+1)(2n+1)}{6} + \dfrac{n(n+1)}{2})$$
		$$= \dfrac{n(n+1)}{2}(\dfrac{(2n+1)}{6} + \dfrac{1}{2}) = \dfrac{n(n+1)(2n+4)}{12} = \dfrac{n(n+1)(n+2)}{6}$$
		Хмм... Введём для удобства обобщённую форму записи суммы сумм сумм... Обозначим глубину суммы как d и выразим через рекуренту:
		$$S_k^d(n) = \sum_{j=1}^{n}S_k^{d-1}(j)$$
		Из этого замечу интересный факт о сумме нулевой глубины - это просто размер суммы в степени k:
		$$S_k^0(n) = n^k$$
		Найду сумму для k=1 для других глубин, как я это делал выше и выпишу:
		\begin{align*}
			S_1^0(n) &= n\\
			S_1^1(n) &= \dfrac{n(n+1)}{2}\\
			S_1^2(n) &= \dfrac{n(n+1)(n+2)}{6}\\
			S_1^3(n) &= \dfrac{n(n+1)(n+2)(n+3)}{24}\\
			\vdots
		\end{align*}
		Читается закономерность:
		$$S_1^d(n) = \dfrac{\displaystyle \prod_{j=0}^{d}n+j}{(d+1)!} = \dfrac{(n+d)!}{(n-1)!(d+1)!}$$
		Я не очень понимаю откуда она берётся. Было бы интересно придать этому какой-то геометрический смысл или найти место где она появляется.
	}
	\newpage
	
	\section{Таблица сумм в пространстве натуральных d и k}{	
		
		КАРОЧЕ Я БЫ МОГ ВАМ ЭТО ВСЁ ТУТ МИЛЬОН ЛЕТ РАСПИСЫВАТЬ ОБОБЩЁННУЮ ДЛЯ КАЖДОЙ K, НО Я ЭТО ДЕЛАТЬ НЕ БУДУ ТАК КАК ТАМ НИЧЕГО СЛОЖНОГО НЕТ И РАБОТА РУТИННАЯ, ПРИВЕДУ СРАЗУ РИЗУЛЬТАТ ВВИДИ ТАБЛИЧКИ 
		\begin{figure}[!h]		
			\makebox[1 \textwidth][c]{       %centering table
				\resizebox{1.3 \textwidth}{!}{   %resize table
					\bgroup
					\def\arraystretch{2.5}
					\begin{tabular}{|c|c|c|c|c|c|c|c|} \hline
						d\textbackslash k&0&1&2&3&4&\ldots&k\\ \hline
						0&1&$n$&$n^2$&$n^3$&$n^4$&\ldots&$n^k$\\ \hline
						1&n&$\dfrac{n(n+1)}{2}$&$\dfrac{n(n+1)(2n+1)}{6}$&$\dfrac{n(n+1)(6n^2+6n)}{24}$&$\dfrac{n(n+1)(2n+1)(12n^2+12n-4)}{120}$&\ldots&?\\ \hline
						2&$\dfrac{n(n+1)}{2}$&$\dfrac{n(n+1)(n+2)}{6}$&$\dfrac{n(n+1)(n+2)(2n+2)}{24}$&$\dfrac{n(n+1)(n+2)(6n^2+12n+2)}{120}$&$\dfrac{n(n+1)(n+2)(2n+2)(12n^2+24n-6)}{720}$&\ldots&?\\ \hline
						3&$\dfrac{n(n+1)(n+2)}{6}$&$\dfrac{n(n+1)(n+2)(n+3)}{24}$&$\dfrac{n(n+1)(n+2)(n+3)(2n+3)}{120}$&$\dfrac{n(n+1)(n+2)(n+3)(6n^2+18n+6)}{720}$&$\dfrac{n(n+1)(n+2)(n+3)(2n+3)(12n^2+36n-6)}{5040}$&\ldots&?\\ \hline
						4&$\dfrac{n(n+1)(n+2)(n+3)}{24}$&$\dfrac{n(n+1)(n+2)(n+3)(n+4)}{120}$&$\dfrac{n(n+1)(n+2)(n+3)(n+4)(2n+4)}{720}$&$\dfrac{n(n+1)(n+2)(n+3)(n+4)(6n^2+24n+12)}{5040}$&$\dfrac{n(n+1)(n+2)(n+3)(n+4)(2n+4)(12n^2+48n-4)}{40320}$&\ldots&?\\ \hline
						\vdots&\vdots&\vdots&\vdots&\vdots&\vdots&$\ddots$&\vdots\\ \hline
						d&$\dfrac{(n+d)!\dfrac{1}{n+d}}{(n-1)!d!}$&$\dfrac{(n+d)!}{(n-1)!(d+1)!}$&$\dfrac{(n+d)!(2n+d)}{(n-1)!(d+2)!}$&$\dfrac{(n+d)!(6n^2+6dn+d(d-1))}{(n-1)!(d+3)!}$&$\dfrac{(n+d)!(2n+d)(12n^2+12dn+d(d-5))}{(n-1)!(d+4)!}$&\ldots&\textbf{?}\\ \hline
					\end{tabular}
					\egroup
				} %close resize
			} %close centering		
		\end{figure}
		
		ПРОСТА КАК ЗАМЕЧАНИЕ, ЭТИ ПРИКОЛЫ ПОХОЖИ НА БИНОМИАЛЬНЫЕ КОЭФИЦЕНТЫ НЕ ЗНАЮ ЧТО С ЭТИМ ФАКТОМ ДЕЛАТЬ
		$${n \choose k} = \dfrac{n!}{k!(n-k)!}$$
		МОЖНА УВИДИТЬ КРУТУЮ ШТУКУ, ВЕЗДЕ ПОВТОРЯЮТСЯ ФАКТОРИАЛЫ, ОБОБЩЁННЫЕ ФОРМУЛЫ ОТЛИЧАЮТСЯ ТОЛЬКО МНОЖИТЕЛЕМ СТЕПЕНИ K-1, ОБОЗНАЧИМ ЕГО КАК $N_k^d(n)$:
		
		$$S_k^d(n) = \dfrac{(n+d)!}{(n-1)!(k+d)!}N_k^d(n)$$
		$$O(N_k^d(n)) = n^{k-1}$$
		ВОТ ЭТО ДА! КЛАСС! МЫ СУЗИЛИ ЗАДАЧУ, ТЕПЕРЬ ЧТОБЫ ОТВЕТИТЬ НА ПЕРВОНАЧАЛЬНЫЙ ВОПРОС ПРО СУММЫ С ПРОИЗВОЛЬНЫМИ НАТУРАЛЬНЫМ k, d И n НАМ ДОСТАТОЧНО ОТВЕТИТЬ НА ТАКОЙ ЖЕ ВОПРОС ПРО МНОГОЧЛЕНЫ, ВЫПИШУ ПЕРВЫЕ ИЗ НИХ:
		
		\begin{landscape}
			\thispagestyle{empty}
			\begin{figure}[!h]		
				\makebox[1 \textwidth][c]{       %centering table
					\resizebox{1.3 \textwidth}{!}{   %resize table
						\bgroup
						\def\arraystretch{1.5}
						\begin{tabular}{l}
							ПОПЫТАЛСЯ ВЫПИСАТЬ МНОГОЧЛЕН ДЛЯ ПЕРВЫХ K И ПРОИЗВОЛЬНЫМ d, ВИДНЫ КАКИЕ-ТО СВЯЗИ С ФАКТОРИАЛОМ И МНОЖИТЕЛЬ (2N+D) ДЛЯ ЧЁТНЫХ K, НО СЛОЖНО, СРАЗУ ПОНЯТЬ ЗАКОНОМЕРНОСТЬ НЕ ПОЛУЧАЕТСЯ
							$N_0^d(n) = \dfrac{1}{n+d}$\\
							$N_1^d(n) = 1$\\
							$N_2^d(n) = (2n+d)$\\
							$N_3^d(n) = \qquad\qquad(6n(n+d)+d(d-1))$\\
							$N_4^d(n) = (2n+d)(12n(n+d)+d(d-5))$\\
							$N_5^d(n) = \qquad\qquad(120n^3(n + 2d) + 30dn(n(5d-3) + d(d-3)) + d(d^3-16d^2+11d+4))$\\
							$N_6^d(n) = (2n+d)(360n^3(n+2d)+60dn(n(7d-7)+d(d-7))+d(d^3-42d^2+119d+42))$\\
							$N_7^d(n) = \qquad\qquad(5040n^5(n + 3d) + 8400n^3d(n(2d - 1) + d(d-2)) + 42nd(n(43d^3 - 254d^2 + 97d + 34) + d(3d^3 - 54d^2 + 97d + 34)) + d(d-1)(d^4-98d^3+659d^2+518d+120))$\\
							$\vdots$\\
							$N_k^d(n) = ????$\\\\
							ПОПЫТАЛСЯ ВЫПИСАТЬ МНОГОЧЛЕН ДЛЯ ПЕРВЫХ K И ПРОИЗВОЛЬНЫМ d, ВИДНЫ КАКИЕ-ТО СВЯЗИ С ФАКТОРИАЛОМ И МНОЖИТЕЛЬ (2N+D) ДЛЯ ЧЁТНЫХ K, НО СЛОЖНО, СРАЗУ ПОНЯТЬ ЗАКОНОМЕРНОСТЬ НЕ ПОЛУЧАЕТСЯ. ЧТОЖ ВЫПИШУ ТОЖЕ САМОЕ НО БЕЗ РАЗЛОЖЕНИЯ НА МНОЖИТЕЛИ:\\
							$N_0^d(n) = \dfrac{1}{n+d}$\\
							$N_1^d(n) = 1$\\
							$N_2^d(n) = (2n)+(d)$\\
							$N_3^d(n) = (6n^2) + (d^2 - d) + \qquad\qquad\qquad\qquad\qquad\qquad\qquad\qquad\qquad(6nd)$\\
							$N_4^d(n) = (24n^3) + (d^3 - 5d^2) + \qquad\qquad\qquad\qquad\qquad\qquad\qquad\qquad(36n^2d + 14nd^2 - 10nd)$\\
							$N_5^d(n) = (120n^4) + (d^4 - 16d^3 + 11d^2 + 4d) + \qquad\qquad\qquad\qquad\qquad(240n^3d - 90n^2d + 150n^2d^2 - 90nd^2 + 30nd^3)$\\
							$N_6^d(n) = (720n^5) + (d^5 - 42d^4 + 119d^3 + 42d^2) + \qquad\qquad\qquad\qquad(1800n^4d + 1560n^3d^2 - 840n^3d + 540n^2d^3 - 1260n^2d^2 + 62nd^4 - 504nd^3 + 238nd^2 + 84nd)$\\
							$N_7^d(n) = (5040n^6) + (d^6 - 99d^5 + 757d^4 - 141d^3 - 398d^2 - 120d) + (15120n^5d + 16800n^4d^2 - 8400n^4d + 8400n^3d^3 - 16800n^3d^2 + 1806n^2d^4 - 10668n^2d^3 + 4074n^2d^2 + 1428n^2d + 126nd^5 - 2268nd^4 + 4074nd^3 + 1428nd^2)$\\
							$\vdots$\\
							$N_k^d(n) = (k!n^{k-1}) + (d^{k-1}...) + (\frac{k-1}{2}k!n^{k-2}d...)$\\\\
							КАКИЕ ТО НАБРОСКИ УЖЕ ЕСТЬ НО СЛОЖНО, ВЕРНЁМСЯ К РАССМОТРЕНИЮ СУММ ДЛЯ D = 1:\\
							$N_0^1(n) = \dfrac{1}{n+1}$\\
							$N_1^1(n) = 1$\\
							$N_2^1(n) = (2n + 1)$\\
							$N_3^1(n) = (6n^2 + 6n)$\\
							$N_4^1(n) = (24n^3 + 36n^2) + 4(n - 1)$\\
							$N_5^1(n) = (120n^4 + 240n^3) + 60n(n - 1)$\\
							$N_6^1(n) = (720n^5 + 1800n^4) + 720n^2(n - 1) - 120(n - 1)$\\
							$N_7^1(n) = (5040n^6 + 15120n^5) + 8400n^3(n - 1) - 3360n(n - 1)$\\
							$N_8^1(n) = (40320n^7 + 141120n^6) + 100800n^4(n - 1) - 68544n^2(n - 1) + 12096(n - 1)$\\
							$N_9^1(n) = (362880n^8 + 1451520n^7) + 1270080n^5(n - 1) - 1270080n^3(n - 1) + 544320n(n - 1)$\\
							$N_{10}^1(n) = (3628800n^9+16329600n^8)+16934400n^6(n-1) - 22982400n^4(n - 1) + 16934400n^2(n - 1) - 3024000(n - 1)$\\
							$N_{11}^1(n) = (39916800n^{10} + 199584000n^9) + 239500800n^7(n - 1) - 419126400n^5(n - 1)+459043200n^3(n-1)-199584000n(n-1)$\\
							$\vdots$\\
							$N_k^1(n) = k!n^{k-1} + \frac{k-1}{2}k!n^{k-2} + \frac{(k-2)(k-3)}{12}k!n^{k-3} - \frac{(k-2)(k-3)}{12}k!n^{k-4} - \frac{(k+9)(k-5)(k-4)(k-2)}{720}k!n^{k-5} + \frac{(k+9)(k-5)(k-4)(k-2)}{720}k!n^{k-6} + \frac{(k-7)(k-6)(k-4)(k-2)(k^2+10k+45)}{30240}k!n^{k-7} - \frac{(k-7)(k-6)(k-4)(k-2)(k^2+10k+45)}{30240}k!n^{k-8}...$\\
							$N_k^1(n) = k!(n^{k-1} + \frac{k-1}{2}n^{k-2} + \frac{(k-2)(k-3)}{12}n^{k-4}(n-1) - \frac{(k+9)(k-5)(k-4)(k-2)}{720}n^{k-6}(n-1) + \frac{(k-7)(k-6)(k-4)(k-2)(k^2+10k+45)}{30240}n^{k-8}(n-1)-...)$\\
							УЖЕ ЛУЧШЕ, ВИДНЫ ЯВНЫЕ ЗАКОНОМЕРНОСТИ, УЖЕ МОЖНО С ХОРОШЕЙ ТОЧНОСТЬЮ ОЦЕНИТЬ РЕЗУЛЬТАТ, И ЕСЛИ ЗНАТЬ ЗАКОНОМЕРНОСТЬ В КОЭФИЦЕНТАХ С K ТО ОТВЕТ НА ЗАДЧУ НАЙДЁТСЯ, НО НЕ БУДУ ПОКА ВВОДИТЬ НОВЫЙ МНОГОЧЛЕН, ПОПРОБУЮ РАССМОТРЕТЬ ТО ЖЕ САМОЕ ДЛЯ D=2:\\
							$N_0^2(n) = \dfrac{1}{n+2}$\\
							$N_1^2(n) = 1$\\
							$N_2^2(n) = (2n+2)$\\
							$N_3^2(n) = 6n^2+12n+2$\\
							$N_4^2(n) = 24n^3+72n^2+36n-12$\\
							$N_5^2(n) = 120n^4 + 480n^3 + 420n^2 - 120n - 60$\\
							$N_6^2(n) = 720n^5 + 3600n^4 + 4560n^3 - 720n^2 - 1920n + 480$\\
							$N_7^2(n) = 5040n^6 + 30240n^5 + 50400n^4 - 37296n^2 + 6048n + 6048$\\
							$\vdots$\\
							$N_k^2(n) = k!(n^{k-1} + (k-1)n^{k-2}...)$\\
						\end{tabular}
						\egroup
					} %close resize
				} %close centering		
			\end{figure}
		\end{landscape}
	}
\end{document} % конец документа