\documentclass[a4paper,12pt]{article} % тип документа (book, report)

%  Русский язык
\usepackage[warn]{mathtext}         % кирилица в формулах
\usepackage[T2A]{fontenc}			% кодировка
\usepackage[utf8]{inputenc}			% кодировка исходного текста
\usepackage[english,russian]{babel}	% локализация и переносы
\usepackage[thinlines]{easytable}   % таблицы

\usepackage{amsmath,amsfonts,amssymb,amsthm,mathtools} % Математика
\usepackage{wasysym}   % Шрифты
\usepackage{graphicx}  % графика
%\usepackage{fancyhdr} % Колонитулы
%\pagestyle{fancy}     % Колонитулы

%Заговолок
\title{Математический кружок МЦНМО, 8 класс. Решения.}
\author{Богдан Нелимов}
\date{\today}


\begin{document} % начало документа
	
	\maketitle
	\newpage
	\section{Прекрасный приветственный листик \\3 октября 2020 г.}
	
	
	\subsection{}{Миша написал на доске в некотором порядке 2020 плюсов и 2019 минусов. Каждую минуту Андрей подходит к доске, стирает любые два знака и пишет вместо них один, причём если он стёр одинаковые знаки, то вместо них он пишет плюс, а если разные — минус. В итоге на доске остался только один знак. Какой?\\
		
		Из условия примём, что порядок стирания и используемые способы не имеет значения\footnote{Так можно делать? Можно ли как-то без этого допущения? Оно дано в условии, но такое работает не во всех задачах.}. Значит 2020 плюсов стиранием $++$ сокращается до $+$. 2019 минусов стиранием $--$ сокращается до $+-$. В итоге у нас остаётся $++-$, которое сокращается до $-$. \\ Можем обобщить этот случай. Примём, что ответ в задаче не меняется при $n+1$ плюсов, $n$ минусов и $n\in\{\mathbb N | n \geqslant 1\}$. Значит если плюсов на 1 больше чем минусов, то всегда ответ будет $-$.
	}
	\subsection{}{У Буратино было 24 золотые монеты, 26 серебряных и 25 медных. У Лисы Алисы он может обменять одну золотую и одну серебряную монеты на одну медную, у Кота Базилио — одну серебряную и одну медную на золотую, и у Карабаса-Барабаса — одну медную и одну золотую на серебряную. После многих обменов у Буратино осталась одна монета. Какая?\\
		
		Из условия примём, что порядок обменов и используемые обмены не имеют значения\footnote{Опять же. Хотелось бы без таких допущений.}. Примем эти правила:\\
		$x$, $y$, $z$ - золотая, серебренная, медная.\\
		$x \Leftrightarrow y, z$ - обмен у Кота Базилио.\\
		$y \Leftrightarrow x, z$ - обмен у Карабаса-Барабаса.\\
		$z \Leftrightarrow x, y$ - обмен у Лисы Алисы.\\\\
		Изначально у нас: $x=24; y=26; z=25$. Обменяем монеты у каждого по разу. Получилось $x=23; y=25; z=24$. Каждого вида монет уменьшилось на 1. Значит омбеняв у каждого ещё по 23 раза у нас получится $x=0; y=2; z=1$. А из этого уже просто понять, что останется только $z=1$. То есть медная монета. \\ Можем также обобщить\footnote{Как можно обобщить уже сокращённый случай?} этот случай. Примём, что ответ в задаче не меняется при $\{x=n; y=n+2; z=n+1\}$ и $n\in\{\mathbb N | n \geqslant 0\}$. Значит при наборе монет удовлетворяющих условию ответ всегда будет медная монета.
	}
	\subsection{}{Вера гуляет с собачкой вдоль дорожки, имеющей вид границы прямоугольника со сторонами 7 метров и 5 метров. Любопытная собачка идёт на поводкедлиной 2 метра и может гулять по любую сторону от дорожки. Нарисуйте участок,который может обойти собачка, пока Вера проходит всю дорожку.\\
		\includegraphics[scale=0.3]{"Задача 3"}
	}
	\subsection{}{На торжественной встрече племён Мумба и Юмба каждый человек из племени Мумба подарил по банану двоим людям из племени Юмба, а каждый человек из племени Юмба получил по банану от троих людей из племени Мумба. Во сколько раз людей из племени Мумба было больше, чем из племени Юмба?\\
		
		Каждый из Мумба подарил 2 банана Юмбе\\
		Каждый из Юмба получил 3 банана из Мумба\\
		m - человек в мумба. n - человек в юмба\\
		Чтобы количество подаренных бананов было кратно 3, а каждый при этом подарил бы по 2, человек должно быть кратно 3. Каждые 3 человека в Мумба подарят 6 бананов, их примут 2 человека из Юмбы.\footnote{Простите за странное обоснавание. Тут проще объяснить если порисовать.} $m~\vdots~3; n~\vdots~2; m = \frac{3}{2}n$. То есть людей из племени Мумба в 1.5 раза больше.
	}
	\subsection{}{Министр Маленькой страны хочет соединить все города авиаперелётами так, что каждый город соединён прямым рейсом не более чем с тремя другими, и от каждого города можно долететь в любой другой, сделав не более одной пересадки. Какое максимальное число городов может быть в Маленькой стране?
		
	}
	
	\newpage
	\section{Московская математическая олимпиада. \\28 марта 2021 г.}
	
	\subsection{}{Барон Мюнхгаузен утверждает, что к любому двузначному числу можно справа приписать ещё две цифры так, чтобы получился полный квадрат (к примеру, если задано число $10$, то дописываем $24$ и получаем $1024 = 32^2$). Прав ли барон?\\
		
		Вопрос можно переформулировать так. Если перебирать первые 2 цифры всех 4 значных квадратов, будут ли там все двузнычные числа? То есть n - количество 2 значных чисел и m - количество квадратов дающих 4 значное число. $n = 90; m = 99-31 = 68$. Есть 3 квадрата дающие одинаковые двузначение числа в начале $10, 12 и 16$. Значит $68-3 = 65$. Квадратов меньше чем чисел, значит существует $90-65=25$ чисел с которыми не получится то что утверждает Барон. Также можно просто привести контр пример. $99^2 = 9801; 100^2 = 10000$. Не существует квадрата дающего 99 в начале. Соответствено 99 нельзя дополнить до квадрата.
	}
	
	\subsection{}{Митя купил на день рождения круглый торт диаметром 36 сантиметров и 13 тоненьких свечек. Мите не нравится, когда свечки стоят слишком близко, поэтому он хочет поставить их на расстоянии не меньше 10 сантиметров друг от друга. Поместятся ли все свечки на торте?\\
		
		Для $N>=6$ свечей самым оптимальным способом разместить их будет вершины правильного вписанного в окружность многоугольника с $N-1$ вершинами и 1 свеча посередине торта. То есть минимальным расстоянием между свечами при $N>=6$ свечей и оптимальной расстоновкой будет сторона вписанного многоугольника. Она из формулы $R = \frac{a}{2\sin(\frac{\pi}{N-1})}$ в  $a = D\sin(\frac{\pi}{N-1})$. Для $N = 12$ сторона $a = 36sin(\frac{\pi}{11}) = 10.14$, то есть подходит. Для $N = 13$ сторона $a = 36sin(\frac{\pi}{12}) = 9.32$, то есть не подходит.
	}
	
	\newpage
	\section{Нематематика \\24 апреля 2020 г.}
	
	\subsection{}{В языке Древнего Племени алфавит состоит всего из двух букв: А и Б. Два слова являются синонимами, если одно из другого можно получить при помощи а) исключения буквосочетаний АБ или ББАА, б) добавления в любое место буквосочетания БА. Являются ли синонимами в языке Древнего Племени слова БАА и АББ?\\
		
		Нельзя. Любыми видами изменения слова мы добавляем и убираем одно и то же количество А и Б. В словах БАА и АББ разное количество А и Б. Значит преобразование невозможно.
	}
	
	\subsection{}{Педро Санчес Гарсиа – испанец. Его папу зовут Антонио Санчес Родригес, а маму – Мария Гарсиа Веласкес. Хуан Фернандес Гомес тоже испанец. Как могут звать его родителей?\\
		(А) Федерико Гомес Лопес и Тереса Санчес Фернандес;\\
		(Б) Диего Фернандес Гарсиа и Лусия Санчес Родригес;\\
		(В) Хосе Гомес Фернандес и Хуана Гарсиа Лопес;\\
		(Г) Алехандро Фернандес Лопес и Луиса Гомес Гарсиа;\\
		(Д) Мигель Гарсиа Фернандес и Рамона Фернандес Санчес\\
		
		Хуан - имя. Не даёт нам никакой информации. Фернандес - фамилия отца. Гомес - фамилия матери. Ответ Г. Потому что только этот вариант подходит под шаблон. 
	}
	
	\subsection{}{Даны баскские числительные и их числовые обозначения в перепутанном порядке: berrogeita bi, laurogeita hiru, berrogeita hamasei, hirurogeita hamar, hogeita bost, laurogei, hirurogeita hamazortzi, berrogeita lau, hogeita hamazazpi; \\ 80, 56, 44, 78, 37, 42, 25, 83, 70. \\ Установите правильные соответствия. Запишите по-баскски: 14, 53, 30.\\
		
		Сгруппируем по первому слову: \\
		berrogeita bi, berrogeita hamasei, berrogeita lau; 3\\
		laurogeita hiru, laurogei; 2\\
		hirurogeita hamar, hirurogeita hamazortzi; 2\\
		hogeita bost, hogeita hamazazpi; 2\\\\
		Если брать числа в 10 ричной системе, то получается какая-то билибиерда. Также мне дали подсказку, о том, что в баскском языке 20 ричная система счисления, поэтому переведу все числа в неё и сгруппирую по первой цифре:\	\
		2G, 24, 22;\\
		3I, 3A; \\
		40, 43;\\
		1H, 15;\\
		
		Сразу можем понять, что	berrogeita это 20. Если принять, что laurogei без окончания это число с 0 в первом порядке, то lau должно быть 4. Значит hiru это 3. hamar и hamazortzi это A и I. hogeita это единственная строка с 1 буквенным и 1 цифровым обозначением, а цифра начинающаяся с hama - буквенная. Значит hogeita это 10, а bost это 5.\\
		hog - 1; berro - 2; hiru - 3; lau - 4; bost - 5;\\
		14 это hamalau. 53 это hirurogeita hamahiru. 30 это hogeita hamazortzi.\j]
	}
	
	\subsection{}{В России для записи звучания китайских слов обычно используется одна из двух транскрипций. При изучении китайского языка чаще используется официально принятая в Китае и наиболее распространённая в мире транскрипция на основе латинского алфавита, называемая «пинь’инь». Для передачи китайских слов в русском языке традиционно используется «система Палладия» – транскрипция на основе кириллического алфавита, названная по имени русского китаеведа архимандрита Палладия (Кафа рова, 1817–1878). Ниже приводятся некоторые слоги китайского языка, записанные в латинской транскрипции, и их соответствии в системе Палладия в перепутанном порядке: zha, shan, rou, chang, zheng, ren, chao, shou, sao, сa чжэн, жоу, чао, чжа, cао, чан, жэнь, шоу, шань, ца. Установите правильные соответствия.\\
		
		chao - чао; shou - шоу; sao - cао; chang - чан; shan - шань; сa - ца; zha - чжа; zheng - чжэн; rou - жоу; ren - жэнь;
	}
	
	\subsection{}{Найдите хотя бы одно четырёхзначное число, обладающее следующим свойством: если сумму всех цифр этого числа умножить на произведение всех его чисел, то в результате получится 3990.\\
		
		Разложим на простые множетили. $3990 = 5*2*3*7*19;$ Пусть 19 это сумма цифр числа, а 5,2,3,7 сами цифры числа. $5+2+3+7 = 17$. Не хватает 2 до 19. Можно добавить 1 и 1 без изменения произведения.
		$1*1*2*3*5*7*(1+1+2+3+5+7) = 3990$. Любое число состоящее из 1, 1, 2, 3, 5, 7 подходит. Например 112357. Также можно умножить 2 и 3 и убрать одну единичку и тоже получится: $1*6*5*7*(1+6+5+7) = 3990$. Например подходит число 1567. Шестизначных чисел с таким свойством столько же сколько и перестановок цифр, то есть $\frac{6!}{2}=360$. А четырёхзначных $4!=24$. То есть всего таких чисел существует $360+24=384$. 
	}
	
	\newpage
	\section{Мои приколы.}
	
	\subsection{}{Есть доска с клеточками n на m. В левом верхнем углу доски сидит мышка, в правом нижнем лежит сыр. Мышка может ходить только на 1 клеточку вправо $\rightarrow$ или на 1 клеточку вниз $\downarrow$. Сколькими способами мышка может добраться до сыра?\\
		
		Чтобы мышке добраться до сыра ей всегда нужно будет сделать $n+m$ движений. $n$ движений $\rightarrow$ и $m$ движений $\downarrow$. Это значит, что ответом будет просто количество перестановок движений вправо и вниз. Это биномиальный коэфицент Ньютона $C_{n+m}^{m} = \frac{(n+m)!}{m!n!}$.
	}
	
	\subsection{}{Возьмём ту же задачу с сыром и мышкой, но усложним. Добавим движение вправо вниз $\searrow$. Сколькими способами мышка может добраться?\\
		
		Так же просто решить как в прошлой задаче не получится, так как $\searrow=\rightarrow\downarrow$, а это значит что количество движений не постоянно. Если в прошлой задаче заполнить каждую клетку колличеством возможных способов то получится треугольник паскаля. Значение каждой клеточки равно сумме верхней и левой клеток.\\
		\begin{center}
			\begin{TAB}(e,1cm,1cm){|c|c|c|c|c|c|}{|c|c|c|c|c|c|}
				1 & 1  & 1  & 1  & 1  & 1  \\ 
				1 & 2  & 3  & 4  & 5  & 6  \\  
				1 & 3  & 6  & 10 & 15 & 21 \\
				1 & 4  & 10 & 20 & 35 & 56 \\
				1 & 5  & 15 & 35 & 70 & 126\\
				1 & 6  & 21 & 56 & 126& 252\\
			\end{TAB}
		\end{center}
		\newpage
		Если также заполнить клетки во второй задаче то получится что-то похожее на треугольник Паскаля, но не он. Значение каждой клеточки равно сумме верхней, левой и левой верхней по диагонали клеток. Дальше мой ход мыслей заканчивается и как обобщить ответ на таблицу n на m я не знаю.
		\begin{center}
			\begin{TAB}(e,1cm,1cm){|c|c|c|c|c|c|}{|c|c|c|c|c|c|}
				1 & 1  & 1  & 1  & 1  & 1  \\ 
				1 & 3  & 5  & 7  & 9  & 11  \\  
				1 & 5  & 13 & 25 & 41 & 61 \\
				1 & 7  & 25 & 63 & 129& 231\\
				1 & 9  & 41 & 129& 321& 681\\
				1 & 11 & 61 & 231& 681& 1683\\
			\end{TAB}
		\end{center}
		
		Чтобы избежать непостояннство количества движений попробую посчитать сумму способов с фиксированным количеством $\searrow = z$. Тогда локальное количество способов с фиксированным $z$ будет: $$
		Z(z, m, n) = C_{n+m-z}^{m-z}*С_m^{m-z} = \frac{(n+m-z)!}{(m-z)!n!} * \frac{m!}{(m-z)!z!} = \frac{(n+m-z)!m!}{(m-z)!^2n!z!};\\$$ А ответом будет сумма: $$\sum_{z=0}^{z=min(n, m)} Z(z, m, n)$$
	}
	
	\newpage
	\section{Школа. Геометрия. Вариант 4.\\Ларшин Михаил. }
	
	\subsection{}{Найдите длину отрезка EF и координаты его середины, если E(-5; 2) и F(7; -6).\\
		
		Длина отрезка между $p_1(x_1;y_1)$ и $p_2(x_2;y_2)$ рассчитывается по формуле $\sqrt{(x_2-x_1)^2+(y_2-y_1)^2}$; $L = \sqrt{(7+5)^2+(-6-2)^2} = \sqrt{12^2+8^2} = \sqrt{144+64} = \sqrt{208} = 4\sqrt{13}$.
		
		Сердина отрезка $p_1p_2$ находится по формуле среднего арифметического координат $p_m(\frac{x_1+x_2}{2};\frac{y_1+y_2}{2})$. $p_m(\frac{-5+7}{2};\frac{2-6}{2}) = p_m(1;-2)$
	}
	
	\subsection{}{Составьте уравнение окружности, центр которой находится в точке C(5; -3) и которая проходит через точку N(2; -4).\\
		
		Уравнение окружности для центральной точки $p_c(x_c;y_c)$ и радиуса $R$ это $(x-x_c)^2 + (y-y_c)^2=R^2$. Если окружность проходит через точку, значит расстояние между центром этой окружности и точкой равно радусу, то есть $CN = R$. Значит уравнение окружности это $(x-5)^2+(y+3)^2=3^2+1^2=10$.
	}
	
	\subsection{}{Найдите координаты вершины К параллелограмма ЕFPK, если E(3; -1), F(-3; 3), P(2; -2).\\
		
		В параллелограмме противоположные стороны одинаковой длины и паралельны, это значит, что и векторы их будут одинаковы, значит мы можем вычесть из точки P известный нам вектор $\vec{EF}$ равный $\vec{PK}$ и получить точку K. $\vec{EF} = (-6; 4); K = P - \vec{EF} = (2+6; -2-4) = (8; -6)$.
	}
	\newpage
	\subsection{}{Составьте уравнение прямой, проходящей через точки D(-3; 9) и K(5; -7).\\
		
		Нужно просто координаты в уравнение прямой $y = kx + b$ и решить систему.\\
		$\begin{cases}
			9 = -3k + b;\\
			-7 = 5k + b;
		\end{cases} \Leftrightarrow y = -2x + 3$.
	}
	
	\subsection{}{Найдите координаты точки, принадлежащей оси ординат и равноудалённой от точек А(-5; 2) и В(-3; 6).\\
		
		Если точка p принадлежит оси ординат значит её x координата равна 0. То есть $p(0;y)$. Если точка равноудаленна, то надо составить уравнение длин отрезков.\\
		$\sqrt{5^2+(y-2)^2} = \sqrt{3^2+(y-6)^2};\\
		5^2+(y-2)^2 = 3^2+(y-6)^2;\\
		16+y^2-4y+2^2 = y^2-12y+6^2;\\
		8y = 16;\\
		y = 2;\\	
		p = (0;2)$
	}
	
	\subsection{}{Составьте уравнение прямой, которая параллельна прямой $y = 4x + 9$ И проходит через центр окружности $x^2 + y^2 + 12x + 8y + 50 = 0$.
		
		Прямая паралельная прямой $y=kx+b$ имеет такой же коэфицент k. Так что k = 4. Координаты центра центр окружности можно просто понять, если привести её уравнение к стандартному виду. $x^2 + 12x + 6^2 + y^2 + 8y + 4^2 = 2; (x + 6)^2 + (y + 4)^2 = 2$. Значит центр окружности это p(-6, -4). Осталось решить уравнение $-4 = 4*-6 + b; b = 20$. То есть уравнение прямой это $y = 4x + 20$
	}
	
	\newpage
	\section{Школа. Алгебра. Вариант 4.\\Ларшин Михаил. }
	
	\subsection{}{Решите уравнения: \\ 
		а) $\frac{x+3}{x-3} = \frac{2x+3}{x}; x \neq 0; x \neq 3;\\
		x(x+3) = (x-3)(2x+3);\\
		x^2 + 3x = 2x^2 - 3x - 9;\\
		-x^2 + 6x + 9 = 0;\\
		x^2 - 6x - 9 = 0;\\
		D = b^2 - 4ac = 36 + 36 = 72;\\
		x_{1,2} = \frac{-b \pm \sqrt{D}}{2a} = 3 \pm \frac{\sqrt{72}}{2} = 3 \pm \sqrt{3*3*2} = 3 \pm 3\sqrt{2};
		$\\
		б) $\frac{x^2+14x+24}{x-2} = 0; x \neq 2;\\
		x^2 +14x+24 = 0;\\
		D = b^2 - 4ac = 14^2 - 4*24 = 4(7^2 - 24) = 4*25 = 100;\\
		x_{1,2} = \frac{-b \pm \sqrt{D}}{2a} = -7 \pm \frac{\sqrt{100}}{2} = -7 \pm 5; x_1 = -2; x_2 = -12;
		$
	}
	
	\subsection{}{Решите уравнения: \\
		а)$\frac{2x+7}{x^2+9x+14} + \frac{1}{x^2+3x+2}=\frac{1}{x+1}; x \neq = -1; x \neq -2; x \neq -7;\\
		x^2+9x+14 = (x+2)(x+7); x^2+3x+2 = (x+2)(x+1);\\
		\frac{2x+7}{(x+2)(x+7)} + \frac{1}{(x+2)(x+1)}=\frac{1}{x+1};\\
		\frac{(2x+7)(x+1)}{(x+7)(x+2)(x+1)} + \frac{x+7}{(x+7)(x+2)(x+1)}=\frac{(x+7)(x+2)}{(x+7)(x+2)(x+1)};\\
		(2x+7)(x+1) + (x+7) = (x+7)(x+2);\\
		(2x+7)(x+1) - (x+7)(x+1) = 0;\\
		x(x+1) = 0;\\
		x = 0;
		$\\
		б)$\frac{x-3}{x-2}+\frac{x-2}{x-3} = \frac{5}{2}; x \neq 2; x \neq 3;\\
		\frac{(x-3)^2 + (x-2)^2}{(x-2)(x-3)} = \frac{5}{2};\\
		2((x-3)^2 + (x-2)^2) = 5(x-2)(x-3);\\
		4x^2-20x+26 = 5x^2 - 25x + 30;\\
		x^2 - 5x + 4 = 0;\\
		\begin{cases}
			x_1+x_2 = 5;\\
			x_1x_2 = 4;
		\end{cases} \Leftrightarrow 
		\begin{cases}
			x_1 = 1;\\
			x_2 = 4;
		\end{cases}
		$
		
	}
	
	\subsection{}{Катер прошел 12 км против течения реки и 5 км по течению. При этом он затратил столько времени, сколько ему потребовалось бы, если бы он шел 18 км по озеру. Какова собственная скорость катера, если известно, что скорость течения реки равна 3 км/ч.\\
		$v_{реки} = 3 \frac{км}{ч};\\L_{против} = 12 км;\\L_{по} = 5 км;\\t = \frac{18}{v};\\
		\frac{L_{против}}{v-v_{реки}} + \frac{L_{по}}{v+v_{реки}} = t;\\
		\frac{12}{v-3} + \frac{5}{v+3} = \frac{18}{v};\\
		\frac{12(v+3) + 5(v-3)}{v^2-3^2} = \frac{18}{v};\\
		v(12(v+3) + 5(v-3)) = 18(v^2-3^2);\\
		v^2 - 21v -2*3^4 = 0;
		D = b^2 - 4ac = 7^2 3^2 + 2^3 3^4 = 3^2 11^2;\\
		x = \frac{-b + \sqrt{D}}{2a} = \frac{21 + \sqrt{3^2 11^2}}{2} = \frac{21 + 33}{2} = 27$
	}
	
	
	
	\subsection{}{Первый сплав содержит 5 процентов меди, второй - 14 процентов меди. Масса второго сплава больше массы первого на 7 кг. Из этих двух сплавов получили третий сплав, содержащий 13 процентов меди. Найдите массу третьего сплава. Ответ дайте в килограммах.\\
		$с_1 = 0.05;\\
		c_2 = 0.14;\\
		m_2 = m_1 + 7;\\
		c_3 = 0.13;\\
		m_3 - ?\\\\
		c_3 = \frac{c_1m_1 + c_2m_2}{m_1 + m_2};\\
		0.13 = \frac{0.05m_1 + 0.14(m_1+7)}{2m_1 + 7};\\
		13 = \frac{19m_1+2*7^2}{2m_1 + 7};\\
		26m_1 + 13*7 = 19m_1+2*7^2;\\
		m_1 = 1;\\
		m_3 = 1 + 1 + 7 = 9 кг$  
	}
	
	\newpage
	\subsection{}{
		а)$2(x^2-9)+3\sqrt{x^2-9}-5=0; t = \sqrt{x^2-3^2};\\
		2t^2 + 3t - 5 = 0;\\
		D = b^2 - 4ac = 3^2 + 2^3 5 = 7^2;\\
		t = \frac{-b \pm \sqrt{D}}{2a} = \frac{-3 \pm 7}{4}; t_1 = -2.5; t_2 = 1;\\
		-2.5 = \sqrt{x^2-3^2}; x \in \varnothing;\\
		1 = \sqrt{x^2-3^2}; x^2 = 10; x = \pm \sqrt{10} 
		$\\
		б)$\sqrt{\frac{3x+2}{2x-3}}+\sqrt{\frac{2x-3}{3x+2}}=\frac{5}{2}; x \neq \frac{3}{2}; x \neq -\frac{2}{3}; t = \frac{3x+2}{2x-3}\\
		\sqrt{t} + \sqrt{t^-1} - \frac{5}{2} = 0;\\
		\frac{2t+2 - 5\sqrt{t}}{2\sqrt{t}} = 0;\\
		2t - 5\sqrt{t} + 2 = 0;\\
		\begin{cases}
			\sqrt{t_1}+\sqrt{t_2} = \frac{5}{2}\\
			\sqrt{t_1}\sqrt{t_2} = 1\\
		\end{cases} \Leftrightarrow 
		\begin{cases}
			\sqrt{t_1} = 2;\\
			\sqrt{t_2} = \frac{1}{2};
		\end{cases} \Leftrightarrow 
		\begin{cases}
			t_1 = 4;\\
			t_2 = \frac{1}{4};
		\end{cases}\\
		4 = \frac{3x+2}{2x-3}; 4(2x-3) = (3x+2); x_1 = \frac{14}{5};\\
		\frac{1}{4} = \frac{3x+2}{2x-3}; 2x-3 = 12x+8; x_2 = -\frac{11}{10}
		$
	}
	
	\newpage
	\section{Математическая абака \\24 апреля 2020 г.}
	
	\subsection{}{
	}
	
\end{document} % конец документа