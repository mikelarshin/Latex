\documentclass[a4paper,12pt]{article} % тип документа (book, report)

%  Русский язык
\usepackage[warn]{mathtext}         % кирилица в формулах
\usepackage[T2A]{fontenc}			% кодировка
\usepackage[utf8]{inputenc}			% кодировка исходного текста
\usepackage[english,russian]{babel}	% локализация и переносы
\usepackage[thinlines]{easytable}   % таблицы

\usepackage{amsmath,amsfonts,amssymb,amsthm,mathtools} % Математика
\usepackage{wasysym}   % Шрифты
\usepackage{graphicx}  % графика
\usepackage{xcolor}    % цвет фона
\usepackage{pgfplots}  % графики
%\usepackage{fancyhdr} % Колонитулы
%\pagestyle{fancy}     % Колонитулы
\pgfplotsset{compat=1.9}

\pagecolor{black}
\color{white}

%Заговолок
\title{ВЗМШ. Решение вступительной работы по Математике на 2020-2021.}
\author{Ларшин Михаил}
\date{\today}


\begin{document} % начало документа
	
	\maketitle
	\newpage
	
	\subsection*{10. (8 - 10 класс)}{Два города А и В расположены на берегу реки на расстоянии 10 км друг от друга. Пароход может проплыть из А в В и обратно за 1 час. Больше или меньше времени понадобится ему, чтобы проплыть 20 км по озеру?\\
		
		$$S_{оз} = 2S_р$$
		\begin{align*}
			\frac{S_{oз}}{v_п} &\vee \frac{S_р}{v_п + v_р} + \frac{S_р}{v_п - v_р}\\
			\frac{2S_р}{v_п} &\vee \frac{S_р(v_п + v_р)+S_р(v_п - v_р)}{(v_п + v_р)(v_п - v_р)}\\
			\frac{2S_р}{v_п} &\vee \frac{2S_рv_п}{v_п^2 - v_р^2}\\
			\frac{1}{v_п} &\vee \frac{v_п}{v_п^2 - v_р^2}\\
			\frac{1}{v_п^2} &\vee \frac{1}{v_п^2 - v_р^2}\\
			v_п^2 &\wedge v_п^2 - v_р^2\\
			0 &> - v_р^2\\
		\end{align*}
		В конце получился знак больше, но знак неравенства был один раз перевёрнут, ведь больше то выражение у которого знаменатель меньше, значит итоговый ответ будет меньше. Параходу понадобится меньше времени чтобы проплыть по озеру.
	}
	
	\subsection*{11. (7 – 10 класс)}{а) Можно ли занумеровать ребра куба натуральными числами от 1 до 12 так, чтобы для каждой вершины куба сумма номеров ребер, которые в ней сходятся, была одинаковой? \\б) Аналогичный вопрос, если расставлять по ребрам куба числа -6,-5,-4,-3,-2,-1,1,2,3,4,5,6.\\
		
		а) Сначала определим что это должна быть за сумма в каждой вершине, обозначим её как $x$. Чтобы найти $x$, надо найти сумму всех вершин, обозначим это как $8x$. У куба 12 рёбер, к каждому ребру прилегает 2 вершины, то есть все номера рёбер считаются по 2 раза, значит $8x$ это сумма номеров от одного до 12 посчитаная два раза:
		\begin{align*}
			8x &= 2\frac{12(12+1)}{2}\\
			x &= \frac{12(12+1)}{8}\\
			x &= \frac{3*13}{2}\\
			x &= 19.5
		\end{align*}
		Нецелое число. Такое невозможно составить из сложения целых, значит так занумеровать рёбра нельзя.\\
		
		б) Пойдём по той же тактике и получим $x=0$, так как все числа сократятся. Такое число уже можно составить из списка чисел, но не гарантирует нам возможность расставить так числа. Попробуем просто нарисовать:\\
		\includegraphics[scale=0.5]{"Cube Task 11"}\\
		Отлично получилось. Значит ответ можно! Интересной собеностью такого куба является то, что номера противоположных ребёр всегда противоположны по знаку и равны по модулю.
	}
	\subsection*{12. (8 - 11 класс)}{Найдите целые числа $x$ и $y$ такие, что $x>y>0$ и $x^3+7y=y^3+7x$.\\
		
		Упрощаем выражение: \\
		$x^3-y^3=7x-7y$\\
		$(x-y)(x^2+xy+y^2)=7(x-y)$\\
		$x^2+xy+y^2=7$\\
		$(x+y)^2=7+xy$\\
		$xy\geq-7$\\
		$x>y>0$\\
	}
	\subsection*{13. (9 – 11 класс)}{Разложите на множители:\\а) $x^8+x^4+1$ (на три множителя)\\б) $x^5+x+1$ (на 2 множителя)\\
		
		а) $1*1*(x^8+x^4+1)$. Готово! Ладно, понимаю, что не это имеется ввиду. 
		$$x^8-x^7+x^7-x^6+x^6-x^5+x^5+x^4-x^3+x^3-x^2+x^2-x+x+1$$
		$$(x^8-x^7+x^6)+(x^7-x^6+x^5)-(x^5-x^4+x^3)+(x^3-x^2+x)+(x^2-x+1)$$
		$$x^6(x^2-x+1)+x^5(x^2-x+1)-x^3(x^2-x+1)+x(x^2-x+1)+(x^2-x+1)$$
		$$(x^2-x+1)(x^6+x^5-x^3+x+1)$$
		$$(x^2-x+1)(x^6+x^5-x^4+x^4-x^3-x^2+x^2+x+1)$$
		$$(x^2-x+1)((x^6+x^5+x^4)-(x^4+x^3+x^2)+(x^2+x+1))$$
		$$(x^2-x+1)(x^4(x^2+x+1)-x^2(x^2+x+1)+(x^2+x+1))$$
		$$(x^2+x+1)(x^2-x+1)(x^4-x^2+1)$$
		б) $x^5+x+1$
		$$x^5-x^4+x^4-x^3+x^3-x^2+x^2+x+1$$
		$$(x^5+x^4+x^3)+(-x^4-x^3-x^2)+(x^2+x+1)$$
		$$x^3(x^2+x+1)-x^2(x^2+x+1)+(x^2+x+1)$$
		$$(x^3-x^2+1)(x^2+x+1)$$
	}
	\subsection*{14. (8 – 11 класс)}{В равнобедренном треугольнике биссектриса угла при основании равна одной из сторон. Определите углы треугольника.\\
	\includegraphics[scale=0.5]{"Triangle Task 14"}\\
	Выделенный красным треугольник тоже является равнобедренным, так как у него две стороны равны, значит $\angle ADC = \alpha$. Большой треугольник имеет такие же углы, значит $\angle ABC = \frac{\alpha}{2}$. Можем составить уравнение и найти $\alpha$.\\
	$$\alpha + \alpha + \frac{\alpha}{2} = 180$$
	$$2.5\alpha = 180$$
	$$\alpha = 72$$
	Углы треугольника: 72°, 72°, 36°
}
	\subsection*{15. (9 – 11 класс)}{а) Докажите, что $а + \frac{1}{a} \geq 2$ при $а>0$.\\б) Постройте график функции $y = x + \frac{1}{x}$\\\\
		а) $$а -2 + \frac{1}{a} \geq 0$$
		$$\sqrt{а}^2 -2 + \sqrt{\frac{1}{a}}^2 \geq 0$$
		$$(\sqrt{a} - \sqrt{\frac{1}{a}})^2 \geq 0$$
		$$\sqrt{a} - \sqrt{\frac{1}{a}} \geq 0$$
		$$a - 1 \geq 0$$
		$$a \geq 1$$
		б) \\
		\begin{tikzpicture}
			\begin{axis}[
				title = Графичек,
				xlabel = {$x$},
				ylabel = {$y$},
				domain = 0:5,
				ymax = 10,
				samples = 1000,
				legend style = {fill=black,draw=white}	
				]
				\addplot[blue] {x+(1/x)};
				\addplot[red] {2};
				\legend{ 
					$x + \frac{1}{x}$, 
					$2$, 
				};
			\end{axis}
		\end{tikzpicture}
	}
	\subsection*{16. (9 – 11 класс)}{Известно, что $a + b + c < 0$ и что уравнение $ax^2+bx+c=0$ не имеет действительных корней. Определите, какой знак имеет число $с$.\\
		
		Если квадратное уравнение не имеет действительных корней, значит его график не персекает ось $Ox$ и имеет один и тот же знак при любом $x$. $a + b + c < 0$ это квадратное уравнение при $x=1$, значит при любом $x$ это квадратное уравнение меньше $0$. Квадратное уравнение при $x=0$ это $с<0$, значит с отрицательно.
}
	\subsection*{17. (9 – 11 класс)}{Можно ли восстановить треугольник по серединам его сторон? А четырёхугольник? Любой ответ требует доказательства!\\
	\includegraphics[scale=0.5]{"Triangle Task 17"}\\
	Для треугольника ответ да. Так как вся информация для построения треугольника у нас есть, мы строим треугольник по трём точкам-серединам и откладываем от каждой его точки в две стороны противоположную от точки сторону. Для каждой стороны у нас есть нужная информация, длина стороны это длина противоположной стороны умноженная на два, а наклон это наклон противоположной стороны. Это следует из теоремы о средней линии треугольника.\\
	
	Для четуёхугольника существует бесконечное множество решений\\
	
	}
	
\end{document} % конец документа